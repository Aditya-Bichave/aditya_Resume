\documentclass[10pt]{article}

%packages

\usepackage{titlesec}
\usepackage{titling}
\usepackage[margin=0.5in] {geometry}
\usepackage{multicol}
\usepackage{hyperref}
\usepackage{graphicx}


%section

\titleformat{\section}
{ \bfseries \uppercase}
{\hspace{-.25in}\thesection}
{0em}
{. }[\titlerule]


%sub Section

\titleformat{\subsection}
{\bfseries\large}
{$\bullet$}
{0em}
{ }

%sub Sub Section

\titleformat{\subsubsection}[runin]
{\bfseries}
{}
{0em}
{-}  

\titlespacing{\subsubSection}
{0em}{0em}{0em}


%make titile

\renewcommand{\maketitle}{
\begin{center}
{\LARGE\bfseries
\theauthor}

\vspace{.25em}
\line(1,0){250}
\end{center}
\begin{multicols}{2}
\noindent
%address
Fl.No. 11/A,\\
Hariangan Society,\\
Near New Life Hospital,\\
Gurudwara Chowk,\\
Akurdi,Pune -411044\\

\columnbreak
\begin{flushright}
Contact: +91-7798497481\\
E-mail Id: aditya.bichave@gmail.com\\
Github: \url{https://github.com/Aditya-Bichave}
\end{flushright}
\noindent
\hspace{10em}
\includegraphics[width = 8em]{Aditya.jpg}

\end{multicols}



}





















\begin{document}

\title{Resume}
\author{Aditya Bichave}
\maketitle

\section{Objective}

To seek a challenging position as a Machine Learning Engineering with an organization of repute, where I can utilize my skills and knowledge of Computer Science concepts and advance technologies like Deep Learning, Image processing.

\section{Education}

\begin{tabular}[8pt]{| c | c | c | c | c |}
\hline
	Degree & College/School & University & passing year & Pass Percentage\\
\hline
	B.E Computer Engineering & Pimpri Chinchawad College of Engineering ,Pune & SPPU & 2021 & -- \\
\hline
	First Year & Pimpri Chinchawad College of Engineering ,Pune & SPPU & 2018 & 9.48 CGPA \\
\hline
	H.S.C & Ashoka Universal School and Jr.College,Nashik & ISC & 2017 & 83.60 \\
\hline
	S.S.C & Sacred Heart Convent High School, Nashik & S.S.C & 2015 & 91.00\\
\hline
\end{tabular} 

\section{Projects}
	\begin{enumerate}
		\item E-yantra Project Based Competition: Project on Animal home coming. Where animals and Habitates are extracted from image and than image is feed to deep learning model which classifies the animal and maps the animal to there habitate. This data is later transfered to a bot which searches the minimum path for reaching to animal, Pick the animal and place it in respective habitate.
			\begin{itemize}
			\item Languages Used: Python , C, C++
			\item Technologies Used: PyTorch , Atmega 2560, OpenCV, Atmel Studio, 
			
			\end{itemize}
		\item Android Attendace App with Firebase support: Android app which helps manage College Attendace of students.
			\begin{itemize}
			\item Languages Used: Java, XML.
			\item Technologies Used: Android Studio, Firebase by Google. 
			
			\end{itemize}
		\item Airship shooting Game:  A Game based on python.  where player has to control airship movement and the firing way of the airship. Player with maximum Score wins.
			\begin{itemize}
			\item Languages Used: Python.
			\item Technologies Used: PyGame , Visual Studio. 
			
			\end{itemize}
  
	\end{enumerate}


\section{Training \& Internship}
\subsection{Training:}
\begin{enumerate}
\item
\end{enumerate}

\section{Research publications}
	NONE.
\section{Technical Skills}
	\subsection{Languages:}
		\subsubsection{Programming Languages: }
			C, C++, Python, Java.
		\subsubsection{Markup Languages:}
			HTML,CSS,XML
		\subsubsection{Tools}
			PyTorch,Tensorflow,Keras, OpenCV,Android Studio,Atmel Studio,Arduino

\section{Soft Skills}
	\begin{multicols}{2}
	\begin{itemize}
	\item Confident \\
 	\item Problem Solving\\
	\item Motivated\\ 
	\item Curious\\
	\columnbreak
	\item Persistent \\
	\item Teamwork\\
	\item Self-managment\\
	\end{itemize}
	\end{multicols}


\section{Extra-curricular Activities}



\section{co-curricular activities}



\section{Personal Details} 
\begin{itemize}

\item{Father's Name: Anand Bichave}
\item{Mother's NAme: Rashmi Bichave}
\item{Sex: Male}
\item{Date Of Birth: 9th April, 1999}
\item{Nationality: Indian}
\item{Marital Status: Single}

\end{itemize}






\end{document} 